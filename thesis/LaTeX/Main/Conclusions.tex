\section{Conclusions}
\label{chap:conclusions}
In general, predictive models allow us to make more useful and less erroneous decisions: making important decisions without diligent consideration to uncertainties in the budgeting process can lead to unrealistic values, but forecasting with accuracy, on how much damage a successful security breach can cause, is a real challenge for risk managers, especially when multiple assets and associated threat exposure are considered \parencite{Fagade}.
In conclusion, using probabilistic simulation, simplifies the complexity of cost estimation processes.
Indeed, the application of Monte Carlo Simulation to information security investment decisions, allows us to visualise different probabilistic outcomes in view of what might go wrong: given the best case, the worst case and the most likely case scenarios.
The Monte Carlo Method allows us to understand the outcome of scenarios and help in identifying unexpected pattern without necessarily exposing information assets to real threats.
It is expected that predictive models will help management making more effective decisions: if there is an effective understanding of what might go wrong, decision makers can utilise this probabilistic model to implement appropriate risk mitigation strategies and budget allocation for security investment, as discussed in \hyperref[chap:risk_management]{Chapter \ref*{chap:risk_management}}.
Moreover, since cybercriminals are becoming really sophisticated and motivated, it becomes increasingly important for computer systems to be protected using advanced intrusion detection systems which are capable of detecting modern malwares \parencite{Khraisat}.